%%%%%%%%%%%%%%%%%%%%%%%%%%%%%%%%%%%%%%%%%%%%%%%%%%%%%%%%%%%%%%%%%%%%%%
%
%.IDENTIFICATION $Id: template.tex.src,v 1.41 2008/01/25 10:47:12 fsogni Exp $
%.LANGUAGE       TeX, LaTeX
%.ENVIRONMENT    ESOFORM
%.PURPOSE        Template application form for ESO Observing time.
%.AUTHOR         The Esoform Package is maintained by the Observing
%                Programmes Office (OPO) while the background software
%                is provided by the User Support System (USS) Department.
%
%-----------------------------------------------------------------------
%
%
%                   ESO LA SILLA PARANAL OBSERVATORY
%                   --------------------------------
%                   NORMAL PROGRAMME PHASE 1 TEMPLATE
%                   ---------------------------------
%
%
%
%          PLEASE CHECK THE ESOFORM USERS' MANUAL FOR DETAILED 
%              INFORMATION AND DESCRIPTIONS OF THE MACROS. 
%     (see the file usersmanual.tex provided in the ESOFORM package) 
%
%
%        ====>>>> TO BE SUBMITTED THROUGH WEB UPLOAD  <<<<====
%               (see the Call for Proposals for details)
%
%%%%%%%%%%%%%%%%%%%%%%%%%%%%%%%%%%%%%%%%%%%%%%%%%%%%%%%%%%%%%%%%%%%%%%

%%%%%%%%%%%%%%%%%%%%%%%%%%%%%%%%%%%%%%%%%%%%%%%%%%%%%%%%%%%%%%%%%%%%%%
%
%                      I M P O R T A N T    N O T E
%                      ----------------------------
%
% By submitting this proposal, the Principal Investigator takes full
% responsibility for the content of the proposal, in particular with
% regard to the names of CoI's and the agreement to act in accordance
% with the ESO policy and regulations, should observing time be
% granted.
%
%%%%%%%%%%%%%%%%%%%%%%%%%%%%%%%%%%%%%%%%%%%%%%%%%%%%%%%%%%%%%%%%%%%%%% 

%
%    - LaTeX *is* sensitive towards upper and lower case letters.
%    - Everything after a `%' character is taken as comments.
%    - DO NOT CHANGE ANY OF THE MACRO NAMES (words beginning with `\')
%    - DO NOT INSERT ANY TEXT OUTSIDE THE PROVIDED MACROS
%

%
%    - All parameters are checked at the verification or submission.
%    - Some parameters are also checked during the pdfLaTeX
%      compilation.  If this is not the case, this is indicated by the
%      phrase
%      "This parameter is NOT checked at the pdfLaTeX compilation."
%

\documentclass{esoform}

% The list of LaTeX definitions of commonly used astronomical symbols
% is already included in the style file common2e.sty (see Table 1 in
% the Users' Manual).  If you have your own macros or definitions,
% please insert them here, between the \documentclass{esoform}
% and the \begin{document} commands.
%
%     PLEASE USE NEITHER YOUR OWN MACROS NOR ANY TEX/LATEX MACROS  
%       IN THE \Title, \Abstract, \PI, \CoI, and \Target MACROS.
%
% WARNING: IT IS THE RESPONSIBILITY OF THE APPLICANTS TO STAY WITHIN THE
% CURRENT BOX LIMITS AND ELIMINATE POTENTIAL OVERFILL/OVERWRITE PROBLEMS 

\begin{document}

%%%%%%%%%%%%%%%%%%%%%%%%%%%%%%%%%%%%%%%%%%%%%%%%%%%%%%%%%%%%%%%%%%%%%%%%
%%%%% CONTENTS OF THE FIRST PAGE %%%%%%%%%%%%%%%%%%%%%%%%%%%%%%%%%%%%%%%
%%%%%%%%%%%%%%%%%%%%%%%%%%%%%%%%%%%%%%%%%%%%%%%%%%%%%%%%%%%%%%%%%%%%%%%%
%
%---- BOX 1 ------------------------------------------------------------
%
% You should use this template for period 95A applications ONLY.
%
% DO NOT EDIT THE MACRO BELOW. 

\Cycle{95A}

% Type below, within the curly braces {}, the title of your observing
% programme (up to two lines).
% This parameter is NOT checked at the pdfLaTeX compilation.
%
% DO NOT USE ANY TEX/LATEX MACROS IN THE TITLE

\Title{Ultra-Deep Ks-band imaging of the final two HST Frontier Fields}  

% Type below the numeric code corresponding to the subcategory of your
% programme.

\SubCategoryCode{A8}   

% Please specify the type of programme you are submitting. 
% Valid values: NORMAL, GTO, TOO, CALIBRATION, MONITORING
% If you specify TOO, you will also need to fill a ToO page below.
% If you specify CALIBRATION, then the SubCategory Code MUST be set to L0

% If your programme requires more than 100 hours the Large Programme
% template (templatelarge.tex) must be used.


\ProgrammeType{NORMAL}

% For GTO proposals only: uncomment the following and fill out the GTO
% programme code (as communicated to the respective GTO coordinator).

%\GTOcontract{INS-consortium}		

% For TOO proposals only: uncomment the following if you apply for
% Rapid Response Mode observations.
 
%\ObservationInRRM{}

% Uncomment the following macro if this proposal is applying for time
% under the VLT-XMM agreement (only available for odd periods).

%\ObservationWithXMM{}

%---- BOX 2 ------------------------------------------------------------
%
% Type below a concise abstract of your proposal (up to 9 lines).
% This parameter is NOT checked at the pdfLaTeX compilation.
%
% DO NOT USE ANY TEX/LATEX MACROS IN THE ABSTRACT

\Abstract{The \textit{HST} Frontier Fields (HFF) programme is a multi-cycle (2013--2015) campaign 
with the Hubble Space Telescope that will image six deep fields centered on strong lensing 
galaxy clusters in parallel with six deep blank fields.  Here we 
propose to extend our P92 \textit{VLT} legacy programme and image the two  final Frontier Fields targets---Abell S1063 and Abell 370---with HAWK-I/$K_s$ to a depth reaching $K_s=26.5$ (AB, 5$\sigma$).  The $K_s$ band at 2.2\,$\mu\mathrm{m}$ crucially fills the gap between 
the reddest \textit{HST} filter (1.6\,$\mu\mathrm{m}$~$\sim$~$H$) and the 
IRAC 3.6\,$\mu\mathrm{m}$ passband, greatly improving the constraints on both the redshifts and the stellar-population 
properties of galaxies extending well below the characteristic stellar mass across most 
of the age of the universe, down to, and including, the redshifts of the targeted galaxy 
clusters ($z<0.5$).  We waive the proprietary rights to the HAWK-I observations in 
order to make this valuable dataset immediately available to the community.
}

%---- BOX 3 ------------------------------------------------------------
%
% Description of the observing run(s) necessary for the completion of
% your programme.  The macro takes ten parameters: run ID, period,
% instrument, time requested, month preference, moon requirement,
% seeing requirement, transparency requirement, observing mode and 
% run type.
%
% 1. RUN ID
% Valid values: A, B, ..., Z
% Please note that only one run per intrument is allowed for APEX
%
% 2. PERIOD
% Valid values: 95
% Exceptions:
% Monitoring Programmes: These programmes can span up to four periods.
%
% VLT-XMM proposals: These are only accepted in odd periods and are 
% also valid for the next period.
%
% This parameter is NOT checked at the pdfLaTeX compilation.
%
% 3. INSTRUMENT
% Valid values: ARTEMIS CHAMPP EFOSC2 FLAMES FLASH FORS2 HARPS HAWKI KMOS LABOCA MUSE NACO OMEGACAM SHFI SINFONI SOFI SOFOSC SPHERE SUPERCAM Special3.6 SpecialAPEX SpecialNTT UVES VIMOS VIRCAM VISIR XSHOOTER
% 
% Only Chilean and GTO Programmes are accepted on OMEGACAM.
% No normal programmes on OMEGACAM will be accepted.
% Please note that only a subset of these instruments will be accepted
% for Monitoring Programmes. Please see the Call for Proposals and the
% ESOFORM User Manual for more details.
%
% 4. TIME REQUESTED
% In hours for Service Mode, in nights for Visitor Mode.
% In either case the time can be rounded up to  1 decimal place. 
% This parameter is NOT checked at the pdfLaTeX compilation.
% 
% 5. MONTH PREFERENCE
% Valid values: apr, may, jun, jul, aug, sep, any
%
% 6. MOON REQUIREMENT
% Valid values: d, g, n
%
% 7. SEEING REQUIREMENT
% Valid values: 0.4, 0.6, 0.8, 1.0, 1.2, 1.4, n
%
% 8. TRANSPARENCY REQUIREMENT
% Valid values: CLR, PHO, THN
%
% 9. OBSERVING MODE
% Valid values: v, s
%
% 10. RUN TYPE
% Valid values: TOO 
% For all Normal & Calibration Programmes this field should be blank.
% For TOO & GTO Programmes, users can specify TOO runs.
% If the field is left blank a default normal, non-TOO run is assumed.
% If a TOO run is specified please make sure that you fill in the TOO page.

\ObservingRun{A}{95}{HAWKI}{79h}{any}{n}{0.6}{CLR}{s}{}
\ObservingRun{B}{95}{HAWKI}{3h}{any}{n}{0.6}{PHO}{s}{}

%\ObservingRun{A}{95}{FORS2}{4h}{may}{n}{0.8}{PHO}{s}{}
%\ObservingRun{A/alt}{95}{FORS2}{3n=2x1+2H2}{may}{n}{0.8}{PHO}{v}{}
%\ObservingRun{B}{95}{VIMOS}{2n=2x1}{jun}{n}{0.6}{CLR}{v}{}
%\ObservingRun{C}{95}{EFOSC2}{3n}{aug}{n}{0.8}{THN}{v}{}
%\ObservingRun{D}{95}{NACO}{0.4n}{may}{n}{0.8}{THN}{v}{}
%\ObservingRun{E}{95}{VIMOS}{1h}{apr}{n}{1.4}{THN}{s}{}
%\ObservingRun{F}{95}{VIMOS}{1h}{apr}{n}{n}{THN}{s}{}


% Proprietary time requested.
% Valid values: % 0, 1, 2, 6, 12

\ProprietaryTime{0}

%---- BOX 4 ------------------------------------------------------------
%
% Indicate below the telescope(s) and number of nights/hours already
% awarded to this programme, if any.
% This macro is optional and can be commented out.
% It is also NOT checked at the pdfLaTeX compilation.

%\AwardedNights{NTT}{4n in 93.B-1234}

%\AwardedNights{UT4}{82h in 092.A-0472}

% Indicate below the telescope(s) and number of nights/hours still
% necessary, in the future, to complete this programme, if any.
% This macro is optional and can be commented out.
% It is also NOT checked at the pdfLaTeX compilation.

%\FutureNights{UT2}{20h}

%---- BOX 5 ------------------------------------------------------------
%
% Take advantage of this box to provide any special remark  (up to three
% lines). In case of coordinated observations with XMM, please specify
% both the ESO period and the preferred month for the XMM
% observations here.
% This macro is optional and can be commented out.
% It is also NOT checked at the pdfLaTeX compilation.

%\SpecialRemarks{This macro is optional and can be commented out. }
  
%---- BOX 6 ------------------------------------------------------------
% Please provide the ESO User Portal username for the Principal
% Investigator (PI) in the \PI field.
%
% For the Co-I's (CoI) please fill in the following details:
% First and middle initials, family name, the institute code
% corresponding to their affiliation. 
% The corresponding affiliation should be entered for EACH
% Co-I separately in order to ensure the correct details of 
% all Co-I's are stored in the ESO database.
% You can find all institute codes listed according to country
% on the following webpage:
% http://www.eso.org/sci/observing/phase1/countryselect.html
%
% For example, if the Co-I's full name is David Alan William Jones,
% his affiliation is the Observatoire de Paris, Site de Paris, 
% you should write:
% \CoI{D.A.W.}{Jones}{1588}
% Further examples are shown below.
% DO NOT USE ANY TEX/LATEX MACROS HERE
%

\PI{GBRAMMER} 
% Replace with PI's ESO User Portal username.

\CoI{D.}{Marchesini}{1767}
\CoI{I.}{Labb\'e}{1716}
\CoI{A.}{Fontana}{1345}
\CoI{A.}{Muzzin}{1716}
\CoI{E.}{Barker}{1698}
\CoI{A.}{Galametz}{1496}
\CoI{M.}{Stefanon}{1716}
\CoI{S.}{Toft}{1227}
\CoI{K.}{Whitaker}{1516}
\CoI{A.}{van der Wel}{1489}

% Please note: 
% Due to the way in which the proposal receiver system parses
% the CoI macro, the number of pairs of curly brackets '{}'
% in this macro MUST be strictly equal to 3, i.e., the
% number of parameters of the macro. Accordingly, curly
% brackets should not be used within the parameters (e.g.,
% to protect LaTeX signs).
%
% For instance:
% \CoI{L.}{Ma\c con}{1098}
% \CoI{R.}{Men\'endez}{1098}
%
% are valid, while
%
% \CoI{L.}{Ma{\c}con}{1098}
% \CoI{R.}{Men{\'}endez}{1098}
%
% are not. Unfortunately the receiver does not give an
% explicit error message when such invalid forms are
% used in the CoI macro, but the processing of the proposal
% keeps hanging indefinitely.


%%%%%%%%%%%%%%%%%%%%%%%%%%%%%%%%%%%%%%%%%%%%%%%%%%%%%%%%%%%%%%%%%%%%%%%%
%%%%% THE TWO PAGES OF THE SCIENTIFIC DESCRIPTION AND FIGURES %%%%%%%%%%
%%%%%%%%%%%%%%%%%&&&%%%%%%%%%%%%%%%%%%%%%%%%%%%%%%%%%%%%%%%%%%%%%%%%%%%%
%
%---- BOX 7 ------------------------------------------------------------
%
%               THIS DESCRIPTION IS RESTRICTED TO TWO PAGES 
%
%   THE RELATIVE LENGTHS OF EACH OF THE SECTIONS ARE VARIABLE,
%   BUT THEIR SUM (INCLUDING FIGURES & REFS.) IS RESTRICTED TO TWO PAGES
%
% All macros in this box are NOT checked at the pdfLaTeX compilation.

\ScientificRationale{Galaxy formation is one of the major unsolved problems in 
astrophysics: i.e., how do the $\sim$10$^{-6}$ density fluctuations 
inferred from the CMB radiation at 
$z\sim1100$ [1] subsequently evolve into stars, galaxies, and clusters of 
galaxies that we see today? The standard paradigm of structure formation is 
that of hierarchical assembly in a universe dominated by cold dark matter 
(CDM). CDM galaxy formation models postulate that DM haloes form in a 
dissipationless, gravitational collapse, and the stellar content of galaxies 
forms out of gas inside these structures following the radiative cooling of 
baryons [2].

\vskip 0.1cm

\noindent {\bf The \textit{HST} Frontier Fields:} Through a large investment (560 
orbits) of Director's Discretionary (DD) time, the Hubble Space Telescope 
(\textit{HST}) is currently embarking on a multi-cycle campaign to investigate 
the distant universe ($1$$\lesssim$$z$$\lesssim$$12$). The \textit{HST} Frontier Fields (HFF) programme 
has already imaged two deep fields centered on strong lensing galaxy clusters with two deep blank fields in parallel (Fig.~1).  Four more clusters
and deep fields are planned for the next two \textit{HST} cycles, two of which are visible from the \textit{VLT}.  The primary science 
goals of the twelve HFF fields are to 1) reveal the population of galaxies at 
$z$$=$$5$--$10$ that are 10--50 times fainter intrinsically than any presently 
known, 2) solidify our understanding of the stellar masses and star formation 
histories of sub-$L^{\star}$ galaxies, 3) provide the first statistically 
meaningful morphological characterization of star-forming galaxies at $z>5$, 
and 4) to find $z>8$ galaxies magnified by the cluster lensing, with some bright enough to make them accessible to 
spectroscopic follow-up. The HFF will image 12 times the area of the HUDF (total 
areas of $\sim$55~arcmin$^{2}$ and $\sim$135~arcmin$^{2}$ by WFC3/IR and ACS, 
respectively) with the ACS B$_{\rm 435}$V$_{\rm 606}$I$_{\rm 814}$, and the WFC3 
Y$_{\rm 105}$J$_{\rm 125}$H$_{\rm 140}$H$_{\rm 160}$ filters, reaching a 5$\sigma$ 
total magnitude limit of H$_{\rm 160}$=28.7 (AB). Along with \textit{HST}, \textit{Spitzer} is devoting 1000 hours of DD time to image the six Frontier Fields at 3.6$\mu$m \& 4.5$\mu$m with IRAC.

\vspace{0.1cm} \hspace{0.5cm} Whereas the main goal of the HFF is to explore 
the galaxy population in the first billion years of cosmic history, this 
dataset will be unique for its combination of surveyed area and depth for 
studies of galaxy evolution across most of the age of the universe, down to, 
and including, the redshifts of the targeted galaxy clusters 
($z$$\approx$0.3--0.5). Specifically, the relatively large survey area of the HFF will 
allow for the assembly of a large sample of galaxies at $z$$>$$1$, and its 
depth will enable detailed measurements of their stellar populations and 
morphologies. Fig. 2a shows the cumulative number counts as a function 
of redshift in two Frontier Fields clusters observed in our P92 HAWK-I programme: we 
expect more than a thousand galaxies at $z$$>$$1$ and hundreds at $z$$>$$3$ in 
the combined HAWK-I+HFF survey to $K_s$$=$$26.5$ (with thousands more both outside the \textit{HST} area and also detected in the deeper \textit{HST} imaging with robust upper limits in 
$K_s$).}

\ImmediateObjective{
The observations proposed here will provide deep HAWK-I $K_s$-band imaging to 
fill the large gap in the wavelength coverage between the \textit{HST} $H_{160}$ and 
IRAC 3.6\,$\mu\mathrm{m}$ filters of the ongoing \textit{HST} Frontier Fields survey. 
These observations at 2.2\,$\mu\mathrm{m}$ are critical for enabling galaxy 
studies over the full 8--9 Gyr of cosmic history sampled by the HFF with an 
unprecedented combination of depth and area.  The addition of the $K_s$ band 
significantly improves the precision of both photometric redshifts and derived 
properties of galaxies' stellar populations, such as their stellar mass, at 
$z$$>$$2$ and even at $z$$>$$4$ (Fig.~2b).  The HFF+HAWK-I survey will provide a statistically large sample of galaxies (6$\times$ that of the HUDF+HUGS) down to stellar 
masses $\log{(M/M_{\odot})}\sim$8.5(9.5) at $z=1.5$(4.5), i.e., orders of magnitude below the 
characteristic stellar mass of the stellar mass function, 
$M^{\star}\approx$10$^{11}$~M$_{\odot}$ (e.g., [3,4]).  At high redshifts 
$z$$>$$8$, the $K_s$-band photometry will help constrain the Lyman-break 
redshifts and increase the wavelength lever arm for 
measuring the redshift evolution of the rest-frame UV slopes (i.e., dust 
content and/or metallicity) of the first galaxies [5, 6] (see Fig.~2c).

\vspace{0.1cm} \hspace{0.5cm} We will use the combined HFF+HAWK-I dataset to 
address the critical question of whether or not quiescent galaxies are in 
place in significant numbers at $z$$>$$4$ (Fig.~2). Recent deep near-infrared 
surveys have pushed the discovery of ``red sequence'' galaxies to $z\sim2$ 
(e.g., [4,7,8]). Wide-area surveys have discovered a few such galaxies at 
$z$$\sim$4 but only the most massive galaxies can be detected at the existing survey 
depths [9,4,11]. Deep \textit{HST} photometry alone is insufficient to 
robustly characterize red galaxies at $z$$>$$3$ as even the $H$ band lies 
on the UV side of the Balmer/4000\AA\ break at these redshifts [9,10] 
(Fig.~2b); the $K_s$ band is required to select galaxies at rest-frame optical 
wavelengths while avoiding the confusion challenges of the redder IRAC bands. Combined with the HFF fields we observed in P92, the 
proposed programme is 
needed to reduce the uncertainties due to cosmic variance below 50\% for 
$\sim$10$^{10.5}$~M$_{\odot}$ galaxies at $z$$\sim$6. These uncertainties are substantial in the small HUDF area, which results in only loose upper 
limits on the fraction of quiescent galaxies at $z$$>$$4$ [3].

\vspace{0.1cm} \hspace{0.5cm} Our team has 
extensive experience processing deep $K_s$-band observations (including HAWK-I, 
[5] and Fig.~1).  We are committed to providing a valuable 
resource to the community to make the most productive use of a large investment of \textit{VLT} 
time: we waive the proprietary period on the HAWK-I observations, and we will provide a public release of reduced, registered mosaics 
within one year of completing the integration on each of the two survey 
fields, as demonstrated by our P92 programme \texttt{\small (http://gbrammer.github.io/HAWKI-FF/)} and the ``HUGS'' HAWK-I Large Programme led by Co-I A. Fontana [12] \texttt{\small (http://www.astrodeep.eu/HUGS/)}.

\vspace{0.1cm}
{\tiny {\it References:}
\textbf{[1]} Larson, D., et~al. 2011, ApJS, 192, 16 
\textbf{[2]} White, S. D. M., \& Rees, M. J. 1978, MNRAS, 183, 341 
\textbf{[3]} Lundgren, B., et~al. 2014, ApJ, 780, 34 
\textbf{[4]} Muzzin, A., et~al., 2013a, ApJ, 777, 18
\textbf{[5]} Bouwens, R. et~al., 2013, ApJL, 765, 16
\textbf{[6]} Bouwens, R. et~al., 2012, ApJ, 754, 83
\textbf{[7]} Brammer, G. et~al., 2009, ApJL, 706, 173
\textbf{[8]} Brammer, G. et~al., 2011, ApJ, 739, 24
\textbf{[9]} Marchesini, D. et~al., 2010, ApJ, 725, 1277
\textbf{[10]} Brammer, G. \& van Dokkum, P, 2007, ApJL, 718, 73
\textbf{[11]} Straatman, C. et~al. 2014, ApJ, 783, L14
\textbf{[12]} Fontana, A. et~al., 2014, A\&A, in-press
\textbf{[13]} van der Wel, A. et~al., 2014, ApJ, 788, 28
}

% \textbf{XXX} photo-zs, stellar populations, sample the red edge of the Balmer break at $z>3.X$ whereas WFC3 is rest-UV
% 
% also constraints on reionization era galaxies (Brammer 2013, Bouwens 2013).
% 
% \textbf{XXX} Consider forgoing proprietary time to make this a true community dataset.

} 

%
%---- THE SECOND PAGE OF THE SCIENCE CASE CAN INCLUDE FIGURES ----------
%
% Up to ONE page of figures can be added to your proposal.  
% The text and figures of the scientific description must not
% exceed TWO pages in total. 
% If you use color figures, do make sure that they are still readable
% if printed in black and white. Figures must be in PDF or JPEG format.
% Each figure has a size limit of 1MB.
% MakePicture and MakeCaption are optional macros and can be commented out.

%\MakePicture{P95_f1.pdf}{angle=0,width=17cm}
\MakePicture{P95_f1.jpg}{angle=0,width=17cm}
\MakeCaption{Fig.~1. \textit{Left:} Layout of the existing P92 (PI: Brammer) and proposed P95 HAWK-I pointings on the four \textit{HST} Frontier Fields targets visible from the \textit{VLT}.  The $7.5^\prime\times7.5^\prime$ HAWK-I field of view is perfectly suited to provide simultaneous ultra-deep $K_s$-band imaging of both the cluster and parallel \textit{HST} fields.  \textit{Right:} Cutout of the MACS 0416 field (yellow box at left).  The existing and proposed HAWK-I $K_s$-band imaging provide a critical scientific complement to the ultra-deep \textit{HST} survey, with superb image quality ($0\farcs4$ FWHM) and sensitivity reaching $K_s=26.3$ AB (\textbf{total}, 5$\sigma$, point source).}

%\MakePicture{P95_f2.pdf}{angle=0,width=17cm}
\MakePicture{P95_f2.jpg}{angle=0,width=17cm}
\MakeCaption{Fig.~2. \textit{a)} Cumulative number counts $N(>z)$ of sources in the combined deep HAWK-I/$K_s$ + WFC3/IR area (8~arcmin$^2$/field).  Only a few bright $z\sim8$ candidates are seen in each field (see panel c.4), so \textit{all} of the FF clusters must be observed to maximize the high-$z$, $K_s$-detected sample.  \textit{b)} Without $K_s$-band photometry that samples rest-frame optical wavelengths at $z>2.5$, the uncertainty in stellar mass estimates is extremely large ($\sigma\sim1$ dex). \textit{Panels $c_{1-4}$}) Example \textit{HST}+$K_s$ SED fits from the P92 clusters.  $c_1$ is a remarkable evolved galaxy at $z\sim2.8$ with a prominent bulge and an extended grand-design spiral disk. $c_2$ and $c_3$ show significant evolved components only apparent in the $K_s$-band photometry.  $c_4$ is a robust $z\sim8.6$ Lyman-break $Y$-dropout galaxy with an unambiguous $K_s$-band detection ($K_s=26.0$) that constrains its rest-frame UV spectral slope.  }

%\MakeCaption{}

%%%%%%%%%%%%%%%%%%%%%%%%%%%%%%%%%%%%%%%%%%%%%%%%%%%%%%%%%%%%%%%%%%%%%%
%%%%% THE PAGE OF TECHNICAL JUSTIFICATIONS %%%%%%%%%%%%%%%%%%%%%%%%%%%%%
%%%%%%%%%%%%%%%%%%%%%%%%%%%%%%%%%%%%%%%%%%%%%%%%%%%%%%%%%%%%%%%%%%%%%%%%
%
%---- BOX 8 ------------------------------------------------------------
%
% Provide below a careful justification of the requested lunar phase
% and of the requested number of nights or hours.  
% All macros in this box are NOT checked at the pdfLaTeX compilation.

\WhyLunarPhase{These $K_s$-band observations can be obtained at any lunar 
phase.}  

\WhyNights{

As the figure of merit for achieving the goals of this program, we require 
detecting a galaxy with 1/10th of the characteristic stellar mass of the 
stellar mass function at $z$$=$$4.5$ ($\sim$$10^{10}\,M_\odot$). This ensures 
that such galaxies will be detected in sufficient numbers in the volume probed 
by the 4 \textit{HST} ACS+WFC3 pointings covered here with HAWK-I. Scaling the mass 
completeness limits UltraVISTA [4], we require a 5$\sigma$ depth of 
$K_s$$=$$26.5$ (AB; point source). This is $\sim$3 mag deeper than UltraVISTA 
DR1.

\hspace{0.5cm} For NDIT$\times$DIT = 4$\times$15 s exposures, the HAWK-I 
Exposure Time Calculator indicates that the required depth can be reached in 
30 hours on-source, per field.  This estimate assumes airmass=1.2 and seeing 
of $0\farcs6$ in the $V$ band.  While the seeing constraint implies an 
additional expense in terms of observing efficiency, the resulting seeing of 
FWHM$\sim$$0\farcs45$ in the $K_s$ band is necessary to complement
the high-resolution \textit{HST} imaging ($0\farcs18$ in $H_{160}$) of the crowded HFF 
cluster fields.  This image quality is just sufficient to resolve star-forming galaxies at $z$$>$3 in the rest-frame optical ($\langle R_\mathrm{eff}\rangle$$\sim$3 kpc$=$$0\farcs4$ @ 5$\times$$10^{10}$\,$M_\odot$ [13]).  An additional 11 hours per field is required for the 
acquisition and per-exposure overheads.  We request a total allocation of
41$\times$2=82 hours for the full integrations on the final two southern \textit{HST} 
Frontier Fields. Our experience with the P92 allocation of HAWK-I imaging covering 
the first two Frontier Fields clusters (Abell 2744 and MACS 0416, see Figs.~1~
\&~2) confirms the sensitivity and overhead estimates.  With an identical time 
request (41 h / field), the full mosaics reach 5$\sigma$ depths at 
\textit{total} magnitude $K_s = 26.3$.  The modest difference with respect to 
the ETC prediction is primarily the result of flux in the extended wings of the PSF 
beyond that of the Gaussian profile assumed by the ETC.  In the Abell 2744 field with 
15 s DITs, we achieved 29.4 h on-source.  A bright star in the MACS 0416 field 
required shorter DITs and therefore lower observing efficiency, resulting in 25.5 h 
on-source.

\hspace{0.5cm} We readily acknowledge that this proposal represents a 
significant investment of valuable \textit{VLT} time. However, this 
programme represents a unique opportunity for the 
\textit{VLT} to provide a critical and lasting contribution to these forefront 
survey fields that are also being observed for many hundreds of hours with the \textit{Hubble} and 
\textit{Spitzer} Space Telescopes. The requested allocation will probe parameter space 
largely unexplored from the ground and currently impossible to obtain from space. Furthermore, taking 
into account the roughly factor-of-two (0.75 mag) lensing magnification by the massive 
foreground galaxy clusters, the proposed ultra-deep $K_s$-band 
programme is analogous to over $2$$\times$150 hours of integration on $\sim$$2$$\times$1~arcmin$^2$ of blank survey 
fields. While the cluster magnification comes at a cost of a smaller survey 
volume, the faint luminosities probed will otherwise only be accessible with 
future facilities such as the E-ELT and the JWST.
}


\TelescopeJustification{To match the depth and relatively large area of the 
\textit{HST} Frontier Fields, we require a large telescope aperture and an infrared 
instrument with a large field of view (7.5 arcmin).  The field of view of 
\textit{VLT}/HAWK-I is perfectly matched to cover both the cluster and 
parallel areas of the Frontier Fields in a single pointing (Fig.~1). No other 
large 8--10m-class telescope currently provides a NIR imager wide enough to cover 
both the \textit{HST} cluster and parallel fields at once, instantly decreasing their observing efficiency for this programme by a factor of two. The proposed ultra-deep HAWK-I 
coverage of the HFF survey builds on the \textit{VLT}'s pioneering history of obtaining 
ultra-deep $K$-band images of the HUDF with ISAAC (Labb\'e et al. 2003) and 
HAWK-I [5,12], now over a significantly larger survey area ($\sim$225 arcmin$^2$ over the existing P92 and proposed P95 programmes). }


\ModeJustification{The HAWK-I observations we propose are straightforward imaging acquisitions of large survey fields, and the full integrations that require very good and uniform image quality can be most efficiently scheduled in Service Mode throughout P95.  We do not have specific timing constraints, so the observations taken in Service Mode can accommodate the schedule interruptions of AOF-related activities at UT4.}



% Please specify the type of calibrations needed.
% In case of special calibration the second parameter is used to enter 
% specific details.
% Valid values: standard, special
%\Calibrations{special}{Adopt a special calibration}
\Calibrations{standard}{}


%%%%%%%%%%%%%%%%%%%%%%%%%%%%%%%%%%%%%%%%%%%%%%%%%%%%%%%%%%%%%%%%%%%%%%%
%% PAGE OF BOXES 9-10  %%%%%%%%%%%%%%%%%%%%%%%%%%%%%%%%%%%%%%%%%%%%%%%%
%%%%%%%%%%%%%%%%%%%%%%%%%%%%%%%%%%%%%%%%%%%%%%%%%%%%%%%%%%%%%%%%%%%%%%%
%
%---- BOX 9 -- Use of ESO Facilities --------------------------------
%
% This macro is optional and can be commented out.
% It is also NOT checked at the pdfLaTeX compilation.
% LastObservationRemark: Report on the use of the ESO facilities during
%  the last 2 years (4 observing periods). Describe the status of the
%  data obtained and the scientific output generated.

\LastObservationRemark{PI G. Brammer was PI of 2 \textit{VLT} programs since P90:

% 087.A-0514, ``Confirming the Existence of Monster Galaxies at $z\sim3$'', 22 h, X-shooter.  Roughly 7 of the allotted 22 hours were executed in Service Mode.  A paper describing the spectra has been submitted to the \textit{ApJ}: Marsan et~al. (2014, arXiv/1406.0002).
% 
% 288.A-5036, ``Dissecting a star-bursting dwarf galaxy at $z=1.847$ with VLT/X-shooter and a natural magnifying glass'', 1.5 h, X-shooter.  This DDT was observed on the last possible date of target visibility under substandard conditions and the target was not visible in the reduced spectra.
% 
% 089.B-0543, ``The Star-forming Ancestors of Elliptical Galaxies'', 9 h, SINFONI.  The programme was carried over and completed in P90.  Analysis is ongoing.

\vspace{0.5cm}

\textbf{090.A-0215}, ``Testing the Possible Redshift Variation of the Stellar Initial Mass Function with \textit{VLT}/X-shooter'', 6 h, X-shooter. This programme was assigned priority B and no observations were obtained.

\vspace{0.5cm}

\textbf{092.A-0472}, ``Ultra-Deep $K_s$-band imaging of the \textit{HST} Frontier Fields'', 82 h, HAWK-I.  This programme observed the first two FF clusters that were observed with \textit{HST} in 2013/2014, with the \textit{VLT} observations completed in Feb 2014; see Figs.~1\,\&\,2.  The images have been reduced and the mosaics are made freely available to the community \texttt{\small (http://gbrammer.github.io/HAWKI-FF/)}.  The reduced images and weight maps will be submitted to ESO as a Phase 3 data product in Fall 2014, and a survey description paper is in preparation.

}


%
%---- BOX 9a -- ESO Archive ------------------------------------------
%
% Are the data requested in this proposal in the ESO Archive
% (http://archive.eso.org)? If yes, explain the need for new data.
% This macro is NOT checked at the pdfLaTeX compilation.

\RequestedDataRemark{There is 21 h of archival HAWK-I $J$-band coverage of the Abell 370 field from programme 091.A-0108(B).  The ESO proposal information page indicates that the programme was originally allocated 10.4 h, so it is likely that much of the archival data did not satisfy the observing constraints and will be of lesser quality.  In any case, there is no archival \textit{VLT} 2.2 $\mu$m $K_s$-band imaging of either field, which is required to complement the 0.4--1.6 $\mu$m \textit{HST} survey.}

%
%---- BOX 9b -- ESO GTO/Public Survey Programme Duplications---------
%
% If any of the targets/regions in ongoing GTO Programmes and/or
% Public Surveys are being duplicated here, please explain why.
% This macro is optional and can be commented out.
% It is also NOT checked at the pdfLaTeX compilation.

% \RequestedDuplicateRemark{
%   Specify whether there is any duplication of targets/regions covered 
%   by ongoing GTO and/or Public Survey programmes. If so, please 
%   explain the need for the new data here. Details on the protected 
%   target/fields in these ongoing programmes can be found at: 
% 
%   GTO programmes: http://www.eso.org/sci/observing/teles-alloc/gto.html
%   
%   Public Survey programmes: 
%   http://www.eso.org/sci/observing/PublicSurveys/sciencePublicSurveys.html
%   
%   This macro is optional and can be commented out.
% }

%
%---- BOX 10 ------ Applicant(s) publications ---------------------
%
% Applicant's publications related to the subject of this proposal
% during the past two years.  Use the simplified abbreviations for
% references as in A&A.  Separate each reference with the following
% usual LaTex command: \smallskip\\
%   
%   Name1 A., Name2 B., 2001, ApJ, 518, 567: Title of article1
%   \smallskip\\
%   Name3 A., Name4 B., 2002, A\&A, 388, 17: Title of article2
%   \smallskip\\
%   Name5 A. et al., 2002, AJ, 118, 1567: Title of article3
%
% This macro is NOT checked at the pdfLaTeX compilation.

\Publications{
Bouwens R., et al., 2013, ApJL, 765, 16: ``Photometric Constraints on the Redshift of $z\sim10$ candidate UDFj-39546284 from deeper WFC3/IR+ACS+IRAC observations over the HUDF''
% \smallskip\\
% Brammer, G., et~al., 2011, ApJ, 739, 24: ``The Number Density and Mass Density of Star-forming and Quiescent Galaxies at $0.4 < z < 2.2$''
\smallskip\\
Brammer G., et al., 2013, ApJL, 765, 2: ``A Tentative Detection of an Emission Line at 1.6 $\mu$m for the $z\sim12$ Candidate UDFj-39546284''
% \smallskip\\
% Marchesini, D., et~al., 2012, Apj, 748, 126: ``
% The Evolution of the Rest-frame V-band Luminosity Function from $z = 4$: A Constant Faint-end Slope over the Last 12 Gyr of Cosmic History''
\smallskip\\
Marchesini, D., et~al., 2014, ApJ, 794, 65 (arXiv/1402.0003): ``The Progenitors of Local Ultra-massive Galaxies Across Cosmic Time: from Dusty Star-bursting to Quiescent Stellar Populations''
\smallskip\\
Marsan, C., et~al., 2014, ApJ, submitted (arXiv/1406.0002): ``Spectroscopic Confirmation of an Ultra Massive and Compact Galaxy at $z=3.35$: A Detailed Look at an Early Progenitor of Local Most Massive Ellipticals''
\smallskip\\
Muzzin, A., et~al., 2013, ApJS, 206, 8: ``A Public Ks-selected Catalog in the COSMOS/UltraVISTA Field: Photometry, Photometric Redshifts and Stellar Population Parameters''
\smallskip\\
Muzzin, A., et~al., 2013, ApJ, 777, 18: ``The Evolution of the Stellar Mass Functions of Star-Forming and Quiescent Galaxies to $z = 4$ from the COSMOS/UltraVISTA Survey''
\smallskip\\
Skelton, R., et~al., 2014, ApJ, in-press (arXiv/1403.3689): ``3D-HST WFC3-selected Photometric Catalogs in the Five CANDELS/3D-HST Fields: Photometry, Photometric Redshifts and Stellar Masses''
\smallskip\\
Stefanon, M., et~al., 2013, ApJ, 768, 92: ``What Are the Progenitors of Compact, Massive, Quiescent Galaxies at $z=2.3$? The Population of Massive Galaxies at $z>3$ from NMBS and CANDELS''
\smallskip\\
Stefanon, M., et~al., 2014, ApJ, submitted (arXiv/1408.3416): ``Stellar mass functions of galaxies at $4<z<7$ from an IRAC-selected sample in COSMOS/UltraVISTA: limits on the abundance of very massive galaxies''
\smallskip\\
Straatman, C., et~al., 2014, ApJ, 783, 14: ``A Substantial Population of Massive Quiescent Galaxies at $z\sim4$ from ZFOURGE''
\smallskip\\
van der Wel., A., et~al., 2014, ApJ, 788, 28: ``3D-HST+CANDELS: The Evolution of the Galaxy Size-Mass Distribution since $z = 3$''
% \smallskip\\
% Whitaker, K., et~al., 2011, ApJ, 735, 86: ``The NEWFIRM Medium-band Survey: Photometric Catalogs, Redshifts, and the Bimodal Color Distribution of Galaxies out to $z\sim3$''
}

%%%%%%%%%%%%%%%%%%%%%%%%%%%%%%%%%%%%%%%%%%%%%%%%%%%%%%%%%%%%%%%%%%%%%%%%
%%%%% THE PAGE OF THE TARGET/FIELD LIST %%%%%%%%%%%%%%%%%%%%%%%%%%%%%%%%
%%%%%%%%%%%%%%%%%%%%%%%%%%%%%%%%%%%%%%%%%%%%%%%%%%%%%%%%%%%%%%%%%%%%%%%%
%
%---- BOX 11 -----------------------------------------------------------
%
% Complete list of targets/fields requested.  The macro takes nine
% parameters: run ID, target field/name, RA, Dec, time on target, magnitude, 
% diameter, additional information, reference star.
%
% 1. RUN ID
% Valid values: run IDs specified in BOX 3
%
% 2. TARGET FIELD/NAME
%
% 3. RA (J2000)
% Format: hh mm ss.f, or hh mm.f, or hh.f
% Use 00 00 00 for unknown coordinates
% This parameter is NOT checked at the pdfLaTeX compilation.
% 
% 4. Dec (J2000)
% Format: dd mm ss, or dd mm.f, or dd.f
% Use 00 00 00 for unknown coordinates
% This parameter is NOT checked at the pdfLaTeX compilation.
%
% 5. TIME ON TARGET
% Format: hours (overheads and calibration included)
% This parameter is NOT checked at the pdfLaTeX compilation.
%
% 6. MAGNITUDE
% This parameter is NOT checked at the pdfLaTeX compilation.
%
% 7. ANGULAR DIAMETER
% This parameter is NOT checked at the pdfLaTeX compilation.
%
% 8. ADDITIONAL INFORMATION
% Any relevant additional information may be inserted here.
% For APEX and CRIRES runs, the requested PWV upper limit MUST
% be specified for each target using this field.
% For APEX runs, the acceptable LST range MUST also be specified here.
% This parameter is NOT checked at the pdfLaTeX compilation.
%
% 9. REFERENCE STAR ID
% See Users' Manual.
% This parameter is NOT checked at the pdfLaTeX compilation.
%
% Long lists of targets will continue on the last page of the
% proposal.
%
%                       ** VERY IMPORTANT ** 
% The scheduling of your programme will take into account ALL targets
% given in this list. INCLUDE ONLY TARGETS REQUESTED FOR P95 !
% (except for VLT-XMM proposals)
%
% DO NOT USE ANY TEX/LATEX MACROS FOR THE TARGETS

\Target{AB}{AS1063}{22 48 44}{-44 31 48}{41.0}{26.5}{7 min}{z=0.348 cluster}{}
\Target{AB}{A370}{02 39 53}{-01 34 36}{41.0}{26.5}{7 min}{z=0.375 cluster}{}

%\Target{AB}{A2744}{00 14 21}{-30 23 50}{41.0}{26.5}{7 min}{z=0.308 cluster}{}
%\Target{AB}{MACS0416}{04 16 09}{-24 04 29}{41.0}{26.5}{7 min}{z=0.420 cluster}{}

% \Target{ABC}{Cen A}{13 25 27.61}{-43 01 08.8}{8.0}{7.9}{20 min}{NGC 5128}{}
% \Target{A}{NGC 5139}{13 26.8}{-47 29}{5.0}{6.12}{1 deg}{Omega Cen}{}
% \Target{BC}{NGC 6058}{15 12 51.0}{-38 07 33}{15.0}{11.6}{}{plan. neb.}{}
% \Target{B}{M 5}{15 18 33}{+02 04 58}{8.0}{7}{}{glob. cluster}{}
% \Target{C}{M 6}{17 40.1}{-32 13}{10.0}{2.0}{4.3}{Butterfly cl.}{}
% \Target{C}{M 8}{18 03 37}{-24 23.2}{1.0}{3.8}{30 min}{Lagoon neb.}{}
% \Target{C}{NGC 6822}{19 44 57.8}{-14 48 11}{20.0}{18}{}{Barnard's gal.}{}
% \Target{D}{NGC 7793}{23 57 49.9}{-32 35 20}{20.0}{18}{}{Sd gal.}{S322120026}
% \Target{E}{Alpha Ori}{06 45 08.9}{-16 42 58}{1}{-1.4}{6 mas}{Sirius}{}
% \Target{F}{Alpha Ori}{06 45 08.9}{-16 42 58}{1}{-1.4}{6 mas}{Sirius}{}


%                      ***************** 
%                      ** PWV limits **
% For CRIRES and all APEX instruments users must specify the PWV upper
% limits for each target. For example:
%\Target{}{Alpha Ori}{06 45 08.9}{-16 42 58}{1}{-1.4}{6 mas}{PWV=1.0mm, Sirius}{}
%\Target{}{HD 104237}{12 00 05.6}{-78 11 33}{1}{}{}{PWV<0.7mm;LST=9h00-15h00}{}
%
%                      *****************

% Use TargetNotes to include any comments that apply to several or all
% of your targets.
% This macro is NOT checked at the pdfLaTeX compilation.

% \TargetNotes{A note about the targets and/or strategy of selecting the targets during the run. For APEX runs please remember to specify the PWV limits for each target under 'Additional info' in the table above.}

%%%%%%%%%%%%%%%%%%%%%%%%%%%%%%%%%%%%%%%%%%%%%%%%%%%%%%%%%%%%%%%%%%%%%%%%
%%%%% TWO PAGES OF SCHEDULING REQUIREMENTS %%%%%%%%%%%%%%%%%%%%%%%%%%%%%
%%%%%%%%%%%%%%%%%%%%%%%%%%%%%%%%%%%%%%%%%%%%%%%%%%%%%%%%%%%%%%%%%%%%%%%%
%
%---- BOX 12 -----------------------------------------------------------
%

% Uncomment the following line if the proposal involves time-critical
% observations, or observations to be performed at specific time
% intervals. Please leave these brackets blank. Details of time
% constraints can be entered in Special Remarks and using the
% other flags in Box 13.
%
%
% \HasTimingConstraints{}

%
% The timing constraint macros listed below 
% are optional and can be commented out:
% \HasTimingConstraints, \RunSplitting, \Link and \TimeCritical
% They are also NOT checked at the pdfLaTeX compilation.


% 1. RUN SPLITTING, FOR A GIVEN ESO TELESCOPE (Visitor Mode only)
%
% 1st argument: run ID
% Valid values: run IDs specified in BOX 3
%
% 2nd argument: run splitting requested for sub-runs
% This parameter is NOT checked at the pdfLaTeX compilation.

% \RunSplitting{B}{1,10s,1}
% \RunSplitting{C}{2,10s,2,20w,2,15s,4H2}


% 2. LINK FOR COORDINATED OBSERVATIONS BETWEEN DIFFERENT RUNS.
% \Link{B}{after}{A}{10}
% \Link{C}{after}{B}{}
% \Link{E}{simultaneous}{F}{}

% 3. UNSUITABLE PERIOD(S) OF TIME
%
% 1st argument: run ID
% Valid values: run IDs specified in BOX 3
%
% 2nd argument: Chilean start date for the unsuitable time
% Format: dd-mmm-yyyy
% This parameter is NOT checked at the pdfLaTeX compilation.
%
% 3rd argument: Chilean end date for the unsuitable time
% Format: dd-mmm-yyyy
% This parameter is NOT checked at the pdfLaTeX compilation.

% \UnsuitableTimes{A}{15-jul-15}{18-jul-15}{Insert explanation of unsuitable time here.}
% \UnsuitableTimes{B}{15-jul-15}{18-jul-15}{Insert explanation of unsuitable time here.}
% \UnsuitableTimes{C}{20-jul-15}{23-jul-15}{Insert explanation of unsuitable time here.}


%
%---- BOX 12 contd.. -- Scheduling Requirements 
%

% SPECIFIC DATE(S) FOR TIME-CRITICAL OBSERVATIONS
% Please note: The dates must correspond to the LOCAL CHILEAN observing dates.
%
% The 2nd and 3rd parameters are NOT checked at the pdfLaTeX compilation.
% 1st argument: run ID
% Valid values: run IDs specified in BOX 3
%
% 2nd argument: Chilean start date for the critical period.
% Format: dd-mmmm-yyyy 
%
% 3rd argument: Chilean end date for the critical period.
% Format: dd-mmmm-yyyy

% \TimeCritical{A}{12-may-15}{14-may-15}{Insert reason for time-critical observations.}
% \TimeCritical{D}{1-may-15}{2-may-15}{Insert reason for time-critical observations.}
% \TimeCritical{D}{17-may-15}{18-may-15}{Insert reason for time-critical observations.}
% \TimeCritical{D}{23-may-15}{24-may-15}{Insert reason for time-critical observations.}



%%%%%%%%%%%%%%%%%%%%%%%%%%%%%%%%%%%%%%%%%%%%%%%%%%%%%%%%%%%%%%%%%%%%%%%%
%
%---- BOX 14 -----------------------------------------------------------
%
% INSTRUMENT CONFIGURATIONS:
%
% Uncomment only the lines related to instrument configuration(s)
% needed for the acquisition of your planned observations. 
%
% 1st argument: run ID
% Valid values: run IDs specified in BOX 3
%
% 2nd argument: instrument
% This parameter is NOT checked at the pdfLaTeX compilation.
%
% 3rd argument: mode
% This parameter is NOT checked at the pdfLaTeX compilation.
%
% 4th argument: additional information
% This parameter is NOT checked at the pdfLaTeX compilation.
%
% All parameters are mandatory and cannot be empty. Do NOT specify
% Instrument Configurations for alternative runs.

% Examples (to be commented or deleted)

\INSconfig{A}{HAWKI}{IMG}{Ks}
\INSconfig{B}{HAWKI}{IMG}{Ks}

% \INSconfig{A}{FORS2}{Detector}{MIT} 
% \INSconfig{A}{FORS2}{IMG}{ESO filters: provide list HERE}
% \INSconfig{B}{VIMOS}{IFU 0.33"/fibre}{LR-Blue}
% \INSconfig{C}{EFOSC2}{Imaging-filters}{EFOSC2 filters: provide list here}
% \INSconfig{D}{NACO}{IMG 54 mas/px VIS-WFS}{provide list of filters HERE}
% \INSconfig{E}{VIMOS}{IFU 0.33"/fibre}{LR-Blue}
% \INSconfig{F}{VIMOS}{IFU 0.33"/fibre}{LR-Blue}
%
% Real list of instrument configurations

%%%%%%%%%%%%%%%%%%%%%%%%%%%%%%%%%%%%%%%%%%%%%%%%%%%%%%%%%%%%%%%%%%%%%%%%%
% Paranal
%
%-----------------------------------------------------------------------
%---- NAOS/CONICA at the VLT-UT1 (ANTU)  -------------------------------
%-----------------------------------------------------------------------
%
%\INSconfig{}{NACO}{PRE-IMG}{provide list of filters HERE}
%
% Specify the NGS name, distance from target and magnitude  
%(Vmag preferred, otherwise Rmag) in the target list,
% and uncomment the following line
%\INSconfig{}{NACO}{NGS}{-}
%
%\INSconfig{}{NACO}{Special Cal}{Select if you have special calibrations}
%\INSconfig{}{NACO}{Pupil Track}{Select if you need pupil tracking mode}
%\INSconfig{}{NACO}{Cube}{Select if you need cube mode}
%
%\INSconfig{}{NACO}{SAM VIS-WFS}{Provide list of masks and filters HERE}
%\INSconfig{}{NACO}{SAM IR-WFS}{Provide list of masks and filters HERE}
%\INSconfig{}{NACO}{SAMPol VIS-WFS}{Provide list of masks and filters HERE}
%\INSconfig{}{NACO}{SAMPol IR-WFS}{Provide list of masks and filters HERE}
%
%\INSconfig{}{NACO}{IMG 54 mas/px IR-WFS}{provide list of filters HERE}
%\INSconfig{}{NACO}{IMG 27 mas/px IR-WFS}{provide list of filters HERE}
%\INSconfig{}{NACO}{IMG 13 mas/px IR-WFS}{provide list of filters HERE}
%\INSconfig{}{NACO}{IMG 54 mas/px VIS-WFS}{provide list of filters HERE}
%\INSconfig{}{NACO}{IMG 27 mas/px VIS-WFS}{provide list of filters HERE}
%\INSconfig{}{NACO}{IMG 13 mas/px VIS-WFS}{provide list of filters HERE}
%
%\INSconfig{}{NACO}{CORONA AGPM VIS-WFS}{provide list of filters (L',NB-3.74,NB-4.05) HERE}
%\INSconfig{}{NACO}{CORONA AGPM IR-WFS}{provide list of filters (L',NB-3.74,NB-4.05) HERE}
%
%\INSconfig{}{NACO}{POL 54 mas/px IR-WFS}{provide list of filters HERE}
%\INSconfig{}{NACO}{POL 27 mas/px IR-WFS}{provide list of filters HERE}
%\INSconfig{}{NACO}{POL 13 mas/px IR-WFS}{provide list of filters HERE}
%\INSconfig{}{NACO}{POL 54 mas/px VIS-WFS}{provide list of filters HERE}
%\INSconfig{}{NACO}{POL 27 mas/px VIS-WFS}{provide list of filters HERE}
%\INSconfig{}{NACO}{POL 13 mas/px VIS-WFS}{provide list of filters HERE}
%
%\INSconfig{}{NACO}{APP 54 mas/px IR-WFS}{select Lp and/or NB-4.05}
%\INSconfig{}{NACO}{APP 27 mas/px IR-WFS}{select Lp and/or NB-4.05}
%\INSconfig{}{NACO}{APP 54 mas/px VIS-WFS}{select Lp and/or NB-4.05}
%\INSconfig{}{NACO}{APP 27 mas/px VIS-WFS}{select Lp and/or NB-4.05}
%
%\INSconfig{}{NACO}{SPEC IR-WFS}{provide the list of spectroscopic modes HERE}
%\INSconfig{}{NACO}{SPEC VIS-WFS}{provide the list of spectroscopic modes HERE}
% % 
%

%
%-----------------------------------------------------------------------
%---- FORS2 at the VLT-UT1 (ANTU) --------------------------------------
%-----------------------------------------------------------------------
%If you require the E2V (Blue) detector uncomment the following line
%\INSconfig{}{FORS2}{Detector}{E2V}
%
%If you require the MIT (RED) detector uncomment the following line
%\INSconfig{}{FORS2}{Detector}{MIT}
%
% If you require the High-Resolution  collimator uncomment the following line
%\INSconfig{}{FORS2}{collimator}{HR}
%
% Uncomment the line(s) corresponding to the imaging mode(s) you require and
% provide the list of filters needed  for your observations:
%
%\INSconfig{}{FORS2}{PRE-IMG}{ESO filters: provide list HERE}
%\INSconfig{}{FORS2}{IMG}{ESO filters: provide list HERE}
%\INSconfig{}{FORS2}{IMG}{User's own filters (to be described in text)}
%\INSconfig{}{FORS2}{IPOL}{ESO filters: provide list HERE}
%\INSconfig{}{FORS2}{IPOL}{User's own filters (to be described in text)}
%\INSconfig{}{FORS2}{HIT-MS}{Provide list of grisms HERE}
%
%
% Uncomment the line(s) corresponding to the spectroscopic mode(s) you require and
% provide the list of grism+filter combination needed  for your observations:
%
%\INSconfig{}{FORS2}{LSS}{Provide list of grism+filter combinations HERE}
%\INSconfig{}{FORS2}{MOS}{Provide list of grism+filter combinations HERE}
%\INSconfig{}{FORS2}{PMOS}{Provide list of grism+filter combinations HERE}
%\INSconfig{}{FORS2}{MXU}{Provide list of grism+filter combinations HERE}
%\INSconfig{}{FORS2}{HITI}{ESO filters: provide list HERE}
%\INSconfig{}{FORS2}{HIT-OS}{Provide list of grisms HERE}
%
% Uncomment the following line for Rapid Response Mode observations
%
%\INSconfig{}{FORS2}{RRM}{yes}
%
%-----------------------------------------------------------------------
%---- KMOS at the VLT-UT1 (ANTU) ---------------------------------------
%-----------------------------------------------------------------------
%
%\INSconfig{}{KMOS}{IFU}{provide list of settings (IZ, YJ, H, K, HK) here} 
%
%-----------------------------------------------------------------------
%---- FLAMES at the VLT-UT2 (KUEYEN) -----------------------------------
%-----------------------------------------------------------------------
%\INSconfig{}{FLAMES}{UVES}{Specify the UVES setup below}
%\INSconfig{}{FLAMES}{GIRAFFE-MEDUSA}{Specify the GIRAFFE setup below}
%\INSconfig{}{FLAMES}{GIRAFFE-IFU}{Specify the GIRAFFE setup below}
%\INSconfig{}{FLAMES}{GIRAFFE-ARGUS}{Specify the GIRAFFE setup below}
%\INSconfig{}{FLAMES}{Combined: UVES + GIRAFFE-MEDUSA}{Specify the UVES and
%GIRAFFE setups below}
%\INSconfig{}{FLAMES}{Combined: UVES + GIRAFFE-IFU}{Specify the UVES and
%GIRAFFE setups below}
%\INSconfig{}{FLAMES}{Combined: UVES + GIRAFFE-ARGUS}{Specify the UVES and
%GIRAFFe setups below}
%
%
% If you have selected UVES, either standalone or in combined mode,
% please specify the UVES standard setup(s) to be used
%\INSconfig{}{FLAMES}{UVES}{standard setup Red 520}
%\INSconfig{}{FLAMES}{UVES}{standard setup Red 580}
%\INSconfig{}{FLAMES}{UVES}{standard setup Red 580 + simultaneous calibration}
%\INSconfig{}{FLAMES}{UVES}{standard setup Red 860}
%
%\INSconfig{}{FLAMES}{GIRAFFE}{fast readout mode 625kHz VM only}
%
% If you have selected GIRAFFE, either standalone or in combined mode
% please specify the GIRAFFE standard setups(s) to be used
%\INSconfig{}{FLAMES}{GIRAFFE}{standard setup HR01 379.0}
%\INSconfig{}{FLAMES}{GIRAFFE}{standard setup HR02 395.8}
%\INSconfig{}{FLAMES}{GIRAFFE}{standard setup HR03 412.4}
%\INSconfig{}{FLAMES}{GIRAFFE}{standard setup HR04 429.7}
%\INSconfig{}{FLAMES}{GIRAFFE}{standard setup HR05 447.1 A}
%\INSconfig{}{FLAMES}{GIRAFFE}{standard setup HR05 447.1 B}
%\INSconfig{}{FLAMES}{GIRAFFE}{standard setup HR06 465.6}
%\INSconfig{}{FLAMES}{GIRAFFE}{standard setup HR07 484.5 A}
%\INSconfig{}{FLAMES}{GIRAFFE}{standard setup HR07 484.5 B}
%\INSconfig{}{FLAMES}{GIRAFFE}{standard setup HR08 504.8}
%\INSconfig{}{FLAMES}{GIRAFFE}{standard setup HR09 525.8 A}
%\INSconfig{}{FLAMES}{GIRAFFE}{standard setup HR09 525.8 B}
%\INSconfig{}{FLAMES}{GIRAFFE}{standard setup HR10 548.8}
%\INSconfig{}{FLAMES}{GIRAFFE}{standard setup HR11 572.8}
%\INSconfig{}{FLAMES}{GIRAFFE}{standard setup HR12 599.3}
%\INSconfig{}{FLAMES}{GIRAFFE}{standard setup HR13 627.3}
%\INSconfig{}{FLAMES}{GIRAFFE}{standard setup HR14 651.5 A}
%\INSconfig{}{FLAMES}{GIRAFFE}{standard setup HR14 651.5 B}
%\INSconfig{}{FLAMES}{GIRAFFE}{standard setup HR15 665.0}
%\INSconfig{}{FLAMES}{GIRAFFE}{standard setup HR15 679.7}
%\INSconfig{}{FLAMES}{GIRAFFE}{standard setup HR16 710.5}
%\INSconfig{}{FLAMES}{GIRAFFE}{standard setup HR17 737.0 A}
%\INSconfig{}{FLAMES}{GIRAFFE}{standard setup HR17 737.0 B}
%\INSconfig{}{FLAMES}{GIRAFFE}{standard setup HR18 769.1}
%\INSconfig{}{FLAMES}{GIRAFFE}{standard setup HR19 805.3 A}
%\INSconfig{}{FLAMES}{GIRAFFE}{standard setup HR19 805.3 B}
%\INSconfig{}{FLAMES}{GIRAFFE}{standard setup HR20 836.6 A}
%\INSconfig{}{FLAMES}{GIRAFFE}{standard setup HR20 836.6 B}
%\INSconfig{}{FLAMES}{GIRAFFE}{standard setup HR21 875.7}
%\INSconfig{}{FLAMES}{GIRAFFE}{standard setup HR22 920.5 A}
%\INSconfig{}{FLAMES}{GIRAFFE}{standard setup HR22 920.5 B}
%\INSconfig{}{FLAMES}{GIRAFFE}{standard setup LR01 385.7}
%\INSconfig{}{FLAMES}{GIRAFFE}{standard setup LR02 427.2}
%\INSconfig{}{FLAMES}{GIRAFFE}{standard setup LR03 479.7}
%\INSconfig{}{FLAMES}{GIRAFFE}{standard setup LR04 543.1}
%\INSconfig{}{FLAMES}{GIRAFFE}{standard setup LR05 614.2}
%\INSconfig{}{FLAMES}{GIRAFFE}{standard setup LR06 682.2}
%\INSconfig{}{FLAMES}{GIRAFFE}{standard setup LR07 773.4}
%\INSconfig{}{FLAMES}{GIRAFFE}{standard setup LR08 881.7}
%
%\INSconfig{}{FLAMES}{GIRAFFE}{fast readout mode 625kHz VM only}
%
%-----------------------------------------------------------------------
%---- X-SHOOTER at the VLT-UT2 (KUEYEN)
%-----------------------------------------------------------------------
%
%\INSconfig{}{XSHOOTER}{300-2500nm}{SLT}
%\INSconfig{}{XSHOOTER}{300-2500nm}{IFU}
%
% Slits (SLT only):
%
%UVB arm, available slits in arcsec: 0.5, 0.8, 1.0, 1.3, 1.6, 5.0
%VIS arm, available slits in arcsec: 0.4, 0.7, 0.9, 1.2, 1.5, 5.0 
%NIR arm, available slits in arcsec: 0.4, 0.6, 0.6JH, 0.9, 0.9JH, 1.2, 5.0
%  The 0.6JH and 0.9JH include a stray light K-band blocking filter
%  that allow sky limited studies in J and H bands.
%
%The slits for IFU  are fixed and do not need to be mentioned here.
%
% Replace SLIT-UVB, SLIT-VIS, SLIT-NIR with the choice of the slits:
%\INSconfig{}{XSHOOTER}{SLT}{SLIT-UVB,SLIT-VIS,SLIT-NIR}
%
% Detector readout mode:
%
% UVB and VIS arms: available readout modes and binning:
% 100k-1x1, 100k-1x2, 100k-2x2, 400k-1x1, 400k-1x2, 400k-2x2
% The NIR readout mode is fixed  to NDR.
%
%\INSconfig{}{XSHOOTER}{IFU}{readout UVB,readout VIS,readout NIR}
%\INSconfig{}{XSHOOTER}{SLT}{readout UVB,readout VIS,readout NIR}
%
% Imaging mode 
% replace 'list of filters' by the actual filters you wish to use among:
% U, B, V, R, I, Uprime, Gprime, Rprime, Iprime, Zprime
% Please note that the imaging mode can only be used in combination with slit or IFU observations
%\INSconfig{}{XSHOOTER}{IMG}{list of filters}
%
%\INSconfig{}{XSHOOTER}{RRM}{yes}
%
%-----------------------------------------------------------------------
%---- UVES at the VLT-UT2 (KUEYEN) -------------------------------------
%-----------------------------------------------------------------------
%
%\INSconfig{}{UVES}{BLUE}{Standard setting: 346}
%\INSconfig{}{UVES}{BLUE}{Standard setting: 437}
%\INSconfig{}{UVES}{BLUE}{Non-std setting: provide central wavelength  HERE}
%
%\INSconfig{}{UVES}{RED}{Standard setting: 520}
%\INSconfig{}{UVES}{RED}{Standard setting: 580}
%\INSconfig{}{UVES}{RED}{Standard setting: 600}
%\INSconfig{}{UVES}{RED}{Iodine cell standard setting: 600}
%\INSconfig{}{UVES}{RED}{Standard setting: 860}
%\INSconfig{}{UVES}{RED}{Non-std setting: provide central wavelength HERE}
%
%\INSconfig{}{UVES}{DIC-1}{Standard setting: 346+580}
%\INSconfig{}{UVES}{DIC-1}{Standard setting: 390+564}
%\INSconfig{}{UVES}{DIC-1}{Standard setting: 346+564}
%\INSconfig{}{UVES}{DIC-1}{Standard setting: 390+580}
%\INSconfig{}{UVES}{DIC-1}{Non-std setting: provide central wavelength HERE}
%
%\INSconfig{}{UVES}{DIC-2}{Standard setting: 437+860}
%\INSconfig{}{UVES}{DIC-2}{Standard setting: 346+860}
%\INSconfig{}{UVES}{DIC-2}{Standard setting: 390+860}
%
%\INSconfig{}{UVES}{DIC-2}{Standard setting: 437+760}
%\INSconfig{}{UVES}{DIC-2}{Standard setting: 346+760}
%\INSconfig{}{UVES}{DIC-2}{Standard setting: 390+760}
%\INSconfig{}{UVES}{DIC-2}{Non-std setting: provide central wavelength HERE}
%
%\INSconfig{}{UVES}{Field Derotation}{yes}
%\INSconfig{}{UVES}{Image slicer-1}{yes}
%\INSconfig{}{UVES}{Image slicer-2}{yes}
%\INSconfig{}{UVES}{Image slicer-3}{yes}
%\INSconfig{}{UVES}{Iodine cell}{yes}
%\INSconfig{}{UVES}{Longslit Filters}{Provide list of filters HERE}
%
%\INSconfig{}{UVES}{RRM}{yes}
%
%
%-----------------------------------------------------------------------
%---- SPHERE at the VLT-UT3 (MELIPAL) -----------------------------------
%-----------------------------------------------------------------------
%
%
% Pupil or field tracking?
% Mode choices: IRDIS-CI, IRDIS-DBI, IRDIFS, IRDIFS-EXT, ZIMPOL-I
%               (Not relevant for ZIMPOL-P1 or ZIMPOL-P2)
%--------------------
% IRDIFS: 
% Coronagraph combination choices:
%   IRDIFS:     None, N-ALC-YJH-S, N-ALC-YJH-L, N-CLC-SW-L, N-4Q-YJH
%   IRDIFS-EXT: None, N-ALC-YJH-S, N-ALC-YJH-L, N-ALC-Ks
% Filter choices for IRDIS in IRDIFS mode
%   IRDIFS:     DB-H23, DB-ND23, DB-H34
%   IRDIFS-EXT: DB-K12
%---------------------
% IRDIS imaging (alone):
% Coronagraph combination choices for IRDIS imaging modes (see UM for details)
%   IRDIS-CI:  None, N-ALC-Y, N-ALC-YJ-S, N-ALC-YJ-L, N-ALC-YJH-S, 
%                    N-ALC-YJH-L, N-ALC-Ks, N-4Q-YJH, N-4Q-Ks
%   IRDIS-DBI: None, N-ALC-Y, N-ALC-YJ-S, N-ALC-YJ-L, N-ALC-YJH-S, 
%                    N-ALC-YJH-L, N-ALC-Ks, N-4Q-YJH, N-4Q-Ks
% Filter choices:
%   IRDIS-CI:  BB-Y, BB-J, BB-H, BB-Ks, NB-Hel, NB-CntJ, NB-CntH,
%              NB-CntK1, NB-BrG, NB-CntK2, NB-PaB, NB-FeII, NB-H2, NB-CO
%   IRDIS-DBI: DB-Y23, DB-J23, DB-H23, DB-NDH23,  DB-H34, DB-K12 
%---------------------
% IRDIS spectroscopy:
% Coronagraphic slit/grism combinations for IRDIS-LSS:
%   IRDIS-LSS: N-S-LR-WL, N-S-MR-WL
%---------------------
% ZIMPOL imaging: 
% Coronagraph choices:
%   ZIMPOL-I: None, V-CLC-M-WF, V-CLC-M-NF, V-CLC-L-WF, V-CLC-XL-WF
% Filter choices:
%   ZIMPOL-I: RI, R-PRIM, I-PRIM, V, V-S, V-L, N-R, 730-NB, N-I, I-L,
%             KI,  TiO-717, CH4-727, Cnt748, Cnt820, HeI, OI-630,
%             CntHa, B-Ha, N-Ha, Ha-NB
%--------------------
% ZIMPOL polarimetry:
% Coronagraph choices:
%    ZIMPOL-P1: None, V-CLC-S-WF, V-CLC-M-WF, V-CLC-L-WF, V-CLC-XL-WF, V-CLC-MT-WF
%    ZIMPOL-P2: None, V-CLC-S-WF, V-CLC-M-WF, V-CLC-L-WF, V-CLC-XL-WF, V-CLC-MT-WF
% Filter choices:
%    ZIMPOL-P1: RI, R-PRIM, I-PRIM, V, N-R, N-I, KI, TiO-717, 
%               CH4-727, Cnt748, Cnt820, CntHa, N-Ha, B-Ha     
%    ZIMPOL-P2: RI, R-PRIM, I-PRIM, V, N-R, N-I, KI, TiO-717, 
%               CH4-727, Cnt748, Cnt820, CntHa, N-Ha, B-Ha 
% Readout mode choice for ZIMPOL
%    ZIMPOL-P1: FastPol, SlowPol
%    ZIMPOL-P2: FastPol, SlowPol
%-------------------
%
% One entry per mode. Repeat the entry for each mode.
%
%\INSconfig{}{SPHERE}{Pupil}{mode}
%\INSconfig{}{SPHERE}{Field}{mode}
%
% One entry per combination. Repeat the entry for each combination.
%
%\INSconfig{}{SPHERE}{IRDIFS}{Coronagraph/filter combination for IRDIFS}
%\INSconfig{}{SPHERE}{IRDIFS-EXT}{Coronagraph/filter combination for IRDIFS-EXT}
%
%\INSconfig{}{SPHERE}{IRDIS-CI}{Coronagraph/filter combination for IRDIS-CI}
%\INSconfig{}{SPHERE}{IRDIS-DBI}{Coronagraph/filter combination for IRDIS-DBI}
%
%\INSconfig{}{SPHERE}{IRDIS-LSS}{Coronagraphic slit/grism combination for IRDIS-LSS}
%
%\INSconfig{}{SPHERE}{ZIMPOL-I}{Coronagraph/filter combination for ZIMPOL-I}
%
%\INSconfig{}{SPHERE}{ZIMPOL-P1}{Coronagraph/filter/readout mode for ZIMPOL-P1}
%\INSconfig{}{SPHERE}{ZIMPOL-P2}{Coronagraph/filter/readout mode for ZIMPOL-P2}


%-----------------------------------------------------------------------
%---- VISIR at the VLT-UT3 (MELIPAL) -----------------------------------
%-----------------------------------------------------------------------
%
%\INSconfig{}{VISIR}{IMG N-band 76 mas/px}{Provide list of filters HERE}
%\INSconfig{}{VISIR}{IMG Q-band 76 mas/px}{Provide list of filters HERE}
%
%\INSconfig{}{VISIR}{SPEC N-band LR}{-}
%\INSconfig{}{VISIR}{SPEC N-band HR Longslit}{Provide central wavelengt(s) (8.02,12.81) HERE}
%\INSconfig{}{VISIR}{SPEC Q-band HR Longslit}{Provide central wavelength(s) (17.03) HERE}
%\INSconfig{}{VISIR}{SPEC N-band HRCrossdispersed}{Provide central wavelength(s) (8.99,9.66,10.51,10.52,11.57,11.76,12.27,12.47,12.81-13.30)}
%\INSconfig{}{VISIR}{SPEC Q-band HRCrossdispersed}{Provide central wavelength(s) (16.39,16.92,17.88,17.93,18.24,18.71,21.29) HERE}
%


%-----------------------------------------------------------------------
%---- VIMOS at the VLT-UT3 (MELIPAL) -----------------------------------
%-----------------------------------------------------------------------
%
%\INSconfig{}{VIMOS}{PRE-IMG}{ESO filters: enter the list of filters}
%\INSconfig{}{VIMOS}{IMG}{ESO filters: enter the list of filters}
%\INSconfig{}{VIMOS}{IFU 0.67"/fibre}{LR-Red}
%\INSconfig{}{VIMOS}{IFU 0.67"/fibre}{LR-Blue}
%\INSconfig{}{VIMOS}{IFU 0.67"/fibre}{MR}
%\INSconfig{}{VIMOS}{IFU 0.67"/fibre}{HR-Red}
%\INSconfig{}{VIMOS}{IFU 0.67"/fibre}{HR-Orange}
%\INSconfig{}{VIMOS}{IFU 0.67"/fibre}{HR-Blue}
%
%\INSconfig{}{VIMOS}{IFU 0.33"/fibre}{LR-Red}
%\INSconfig{}{VIMOS}{IFU 0.33"/fibre}{LR-Blue}
%\INSconfig{}{VIMOS}{IFU 0.33"/fibre}{MR}
%\INSconfig{}{VIMOS}{IFU 0.33"/fibre}{HR-Red}
%\INSconfig{}{VIMOS}{IFU 0.33"/fibre}{HR-Orange}
%\INSconfig{}{VIMOS}{IFU 0.33"/fibre}{HR-Blue}
%
%\INSconfig{}{VIMOS}{MOS-grisms}{LR-Red}
%\INSconfig{}{VIMOS}{MOS-grisms}{LR-Blue}
%\INSconfig{}{VIMOS}{MOS-grisms}{MR}
%\INSconfig{}{VIMOS}{MOS-grisms}{HR-Red}
%\INSconfig{}{VIMOS}{MOS-grisms}{HR-Orange}
%\INSconfig{}{VIMOS}{MOS-grisms}{HR-Blue}
%
%\INSconfig{}{VIMOS}{MOS-slits-targets}{0.6" < slit width < 1.4", targets:stellar}
%\INSconfig{}{VIMOS}{MOS-slits-targets}{0.6" < slit width < 1.4", targets:extended}
%\INSconfig{}{VIMOS}{MOS-slits-targets}{slit width > 1.4", targets:stellar}
%\INSconfig{}{VIMOS}{MOS-slits-targets}{slit width > 1.4", targets:extended}
%\INSconfig{}{VIMOS}{MOS-masks}{Enter here number of mask sets (1 set = 4 quadrants)}
%
%
%%-----------------------------------------------------------------------
%---- HAWKI at the VLT-UT4 (YEPUN) -----------------------------------
%-----------------------------------------------------------------------
%
%\INSconfig{}{HAWKI}{PRE-IMG}{provide list of filters (Y,J,H,Ks,CH4,BrG,H2,NB1190,NB1060,NB2090) HERE}
%\INSconfig{}{HAWKI}{IMG}{provide list of filters (Y,J,H,Ks,CH4,BrG,H2,NB1190,NB1060,NB2090) HERE}
%\INSconfig{}{HAWKI}{BURST}{Provide list of filters  (Y,J,H,Ks,CH4,BrG,H2,NB1190,NB1060,NB2090) HERE}
%\INSconfig{}{HAWKI}{FASTJITT}{Provide list of filters  (Y,J,H,Ks,CH4,BrG,H2,NB1190,NB1060,NB2090) HERE}
%\INSconfig{}{HAWKI}{RRM}{yes}
%
%-----------------------------------------------------------------------
%---- SINFONI at the VLT-UT4 (YEPUN) -----------------------------------
%-----------------------------------------------------------------------
%

%\INSconfig{}{SINFONI}{PRE-IMG}{provide list of setting(s) (J,H,K,H+K)}
%
%\INSconfig{}{SINFONI}{IFS 250mas/pix no-AO}{provide list of setting(s) (J,H,K,H+K) HERE}
%\INSconfig{}{SINFONI}{IFS 100mas/pix no-AO}{provide list of setting(s) (J,H,K,H+K) HERE}
%
% If you plan to use a NGS, please specify the NGS name and magnitude (Rmag preferred,
% otherwise Vmag) in target list.
%\INSconfig{}{SINFONI}{IFS 250mas/pix NGS}{provide list of setting(s) (J,H,K,H+K) HERE}
%\INSconfig{}{SINFONI}{IFS 100mas/pix NGS}{provide list of setting(s) (J,H,K,H+K) HERE}
%\INSconfig{}{SINFONI}{IFS 25mas/pix NGS}{provide list of setting(s) (J,H,K,H+K) HERE}
%
% If you plan to use the LGS, please specify the TTS name and magnitude (Rmag preferred,
% otherwise Vmag) in target list.
%\INSconfig{}{SINFONI}{IFS 250mas/pix LGS}{provide list of setting(s) (J,H,K,H+K) HERE}
%\INSconfig{}{SINFONI}{IFS 100mas/pix LGS}{provide list of setting(s) (J,H,K,H+K) HERE}
%\INSconfig{}{SINFONI}{IFS 25mas/pix LGS}{provide list of setting(s) (J,H,K,H+K) HERE}
%
% If you plan to use the LGS without a TTS (seeing enhancer mode) then
% please leave the TTS name blank in the target list.
%\INSconfig{}{SINFONI}{IFS 250mas/pix LGS-noTTS}{provide list of setting(s) (J,H,K,H+K) HERE}
%\INSconfig{}{SINFONI}{IFS 100mas/pix LGS-noTTS}{provide list of setting(s) (J,H,K,H+K) HERE}
%\INSconfig{}{SINFONI}{IFS 25mas/pix LGS-noTTS}{provide list of setting(s) (J,H,K,H+K) HERE}
%
% Select if you have special calibrations
%\INSconfig{}{SINFONI}{Special Cal}{-}
%
% Select if you need pupil tracking mode
%\INSconfig{}{SINFONI}{Pupil Track}{-}
%
% Select for RRM
%\INSconfig{}{SINFONI}{RRM}{yes}
%
%-----------------------------------------------------------------------
%---- MUSE at the VLT-UT4 (YEPUN) -----------------------------------
%-----------------------------------------------------------------------
%
%\INSconfig{}{MUSE}{WFM-NOAO-N}{-}
%\INSconfig{}{MUSE}{WFM-NOAO-E}{-}
%
% Uncomment the following line for Rapid Response Mode observations
%\INSconfig{}{MUSE}{RRM}{yes}
%
%-----------------------------------------------------------------------
%---- VIRCAM at VISTA --------------------------------------------------
%-----------------------------------------------------------------------
%
%\INSconfig{}{VIRCAM}{IMG}{provide list of filters here}
%
%-----------------------------------------------------------------------
%---- OMEGACAM at VST --------------------------------------------------
% This instrument is only available for GTO and Chilean programmes.
%-----------------------------------------------------------------------
%
%\INSconfig{}{OMEGACAM}{IMG}{provide list of filters here}
%
%%%%%%%%%%%%%%%%%%%%%%%%%%%%%%%%%%%%%%%%%%%%%%%%%%%%%%%%%%%%%%%%%%%%%%%%
% La Silla
%-----------------------------------------------------------------------
%---- EFOSC2 (or SOFOSC) at the NTT ------------------------------------
%-----------------------------------------------------------------------
%
%\INSconfig{}{EFOSC2}{PRE-IMG}{EFOSC2 filters: provide list here}
%\INSconfig{}{EFOSC2}{Imaging-filters}{EFOSC2 filters:  provide list here}
%\INSconfig{}{EFOSC2}{Imaging-filters}{ESO non EFOSC filters: provide ESOfilt No}
%\INSconfig{}{EFOSC2}{Imaging-filters}{User's own filters (to be described in text)}
%\INSconfig{}{EFOSC2}{Spectro-long-slit}{Grism\#1:320-1090}
%\INSconfig{}{EFOSC2}{Spectro-long-slit}{Grism\#2:510-1100}
%\INSconfig{}{EFOSC2}{Spectro-long-slit}{Grism\#3:305-610}
%\INSconfig{}{EFOSC2}{Spectro-long-slit}{Grism\#4:409-752}
%\INSconfig{}{EFOSC2}{Spectro-long-slit}{Grism\#5:520-935}
%\INSconfig{}{EFOSC2}{Spectro-long-slit}{Grism\#6:386-807}
%\INSconfig{}{EFOSC2}{Spectro-long-slit}{Grism\#7:327-524}
%\INSconfig{}{EFOSC2}{Spectro-long-slit}{Grism\#8:432-636}
%\INSconfig{}{EFOSC2}{Spectro-long-slit}{Grism\#11:338-752}
%\INSconfig{}{EFOSC2}{Spectro-long-slit}{Grism\#13:369-932}
%\INSconfig{}{EFOSC2}{Spectro-long-slit}{Grism\#14:310-509}
%\INSconfig{}{EFOSC2}{Spectro-long-slit}{Grism\#16:602-1032}
%\INSconfig{}{EFOSC2}{Spectro-long-slit}{Grism\#17:689-876}
%\INSconfig{}{EFOSC2}{Spectro-long-slit}{Grism\#18:470-677}
%\INSconfig{}{EFOSC2}{Spectro-long-slit}{Grism\#19:440-510}
%\INSconfig{}{EFOSC2}{Spectro-long-slit}{Grism\#20:605:715}
%\INSconfig{}{EFOSC2}{Spectro-long-slit}{Aperture: 0.5'', ... ,10.0''}
%
%\INSconfig{}{EFOSC2}{Spectro-long-slit}{Aperture: Shiftable}
%\INSconfig{}{EFOSC2}{Spectro-MOS}{PunchHead=0.95''}
%\INSconfig{}{EFOSC2}{Spectro-MOS}{PunchHead=1.12''}
%\INSconfig{}{EFOSC2}{Spectro-MOS}{PunchHead=1.45''}
%\INSconfig{}{EFOSC2}{Polarimetry}{$\lambda / 2$ retarder plate}
%\INSconfig{}{EFOSC2}{Polarimetry}{$\lambda / 4$ retarder plate}
%\INSconfig{}{EFOSC2}{Coronograph}{yes}
%
%
%-----------------------------------------------------------------------
%---- SOFI (or SOFOSC) at the NTT --------------------------------------------------
%-----------------------------------------------------------------------
%
%\INSconfig{}{SOFI}{PRE-IMG-LargeField}{Provide list of filters HERE}
%\INSconfig{}{SOFI}{Imaging-LargeField}{Provide list of filters HERE}
%\INSconfig{}{SOFI}{Burst}{Provide list of filters HERE}
%\INSconfig{}{SOFI}{FastPhot}{Provide list of filters HERE}
%\INSconfig{}{SOFI}{Polarimetry}{Provide list of filters HERE}
%\INSconfig{}{SOFI}{Spectroscopy-long-slit}{Blue Grism, Provide list of slits HERE}
%\INSconfig{}{SOFI}{Spectroscopy-long-slit}{Red Grism, Provide list of slits HERE}
%\INSconfig{}{SOFI}{Spectroscopy-high-res}{H, Provide list of slits HERE}
%\INSconfig{}{SOFI}{Spectroscopy-high-res}{K, Provide list of slits HERE}
%
%
%-----------------------------------------------------------------------
%---- HARPS at the 3.6 -------------------------------------------------
%-----------------------------------------------------------------------
%
%\INSconfig{}{HARPS}{spectro-Thosimult}{HARPS}
%\INSconfig{}{HARPS}{WAVE}{HARPS}
%\INSconfig{}{HARPS}{spectro-ObjA(B)}{HARPS}
%\INSconfig{}{HARPS}{spectro-ObjA(B)}{EGGS}
%\INSconfig{}{HARPS}{spectro-polarimetry}{linear}
%\INSconfig{}{HARPS}{spectro-polarimetry}{circular}
%
%
%%%%%%%%%%%%%%%%%%%%%%%%%%%%%%%%%%%%%%%%%%%%%%%%%%%%%%%%%%%%%%%%%%%%%%%%
% Chajnantor
%-----------------------------------------------------------------------
%---- SHFI at APEX ----------------------------------------------
%-----------------------------------------------------------------------
%
%\INSconfig{}{SHFI}{APEX-1}{Please enter Central Frequency 211 to 275 GHz}
%\INSconfig{}{SHFI}{APEX-2}{Please enter Central Frequency 275 to 370 GHz}
%\INSconfig{}{SHFI}{APEX-3}{Please enter Central Frequency 385 to 500 GHz} 
%\INSconfig{}{SHFI}{APEX-T2}{Please enter Central Frequency 1.25 to 1.39 THz}
%
%-----------------------------------------------------------------------
%---- LABOCA at APEX ----------------------------------------------
%-----------------------------------------------------------------------
%
%\INSconfig{}{LABOCA}{IMG}{-}
%\INSconfig{}{LABOCA}{PHOT}{-}
%
%-----------------------------------------------------------------------
%---- Artemis at APEX ----------------------------------------------
%-----------------------------------------------------------------------
%\INSconfig{}{ARTEMIS}{IMG}{350 um}
%-----------------------------------------------------------------------
%---- Supercam at APEX ----------------------------------------------
%-----------------------------------------------------------------------
%\INSconfig{}{SUPERCAM}{RECEIVER}{345 GHz}
%
%-----------------------------------------------------------------------
%---- FLASH at APEX ----------------------------------------------
%-----------------------------------------------------------------------
%
%\INSconfig{}{FLASH}{-}{Please enter Central Frequency 272 to 377 GHz and 385 to 495 GHz}
%
%-----------------------------------------------------------------------
%---- CHAMP+ at APEX ----------------------------------------------
%-----------------------------------------------------------------------
%
%\INSconfig{}{CHAMPP}{-}{Please enter Central Frequency 620 to 729 GHz and 780 to 900 GHz}
%-----------------------------------------------------------------------




%%%%%%%%%%%%%%%%%%%%%%%%%%%%%%%%%%%%%%%%%%%%%%%%%%%%%%%%%%%%%%%%%%%%%%%%
%%%%% ToO PAGE %%%%%%%%%%%%%%%%%%%%%%%%%%%%%%%%%%%%%%%%%%%%%%%%%%%%%%%%%
%%%%%%%%%%%%%%%%%%%%%%%%%%%%%%%%%%%%%%%%%%%%%%%%%%%%%%%%%%%%%%%%%%%%%%%%
%
% The \ToOrun macro is needed only when requesting Target of
% Opportunity (ToO) observations, in which case it is MANDATORY to
% uncomment it and fill in the information. It takes the following
% parameters: 
%
% 1st argument: run ID
% Valid values: run IDs specified in BOX 3
%
% 2nd argument: nature of observation
% Valid values: RRM, ToO-hard, ToO-soft
%
% 3rd argument: number of targets per run
% This parameter is NOT checked at the pdfLaTeX compilation.
%
% 4th argument: number of triggers per targets
% (for RRM and ToO observations only)
% This parameter is NOT checked at the pdfLaTeX compilation.

%\TOORun{A}{RRM}{2}{3}
%\TOORun{B}{ToO-hard}{3}{1}

% You have the opportunity to add notes to the ToO runs by using
% the \TOONotes macro.
% This macro is NOT checked at the pdfLaTeX compilation.

%\TOONotes{Use this macro to add a note to the ToO page.}


%%%%%%%%%%%%%%%%%%%%%%%%%%%%%%%%%%%%%%%%%%%%%%%%%%%%%%%%%%%%%%%%%%%%%%%%
%%%%% VISITOR SPECIAL INSTRUMENT PAGE %%%%%%%%%%%%%%%%%%%%%%%%%%%%%%%%%%
%%%%%%%%%%%%%%%%%%%%%%%%%%%%%%%%%%%%%%%%%%%%%%%%%%%%%%%%%%%%%%%%%%%%%%%%
%
% The following commands are only needed when bringing a Visitor
% Special Instrument, in which case it is MANDATORY to uncomment them
% and provide all the required information.
%
%\Desc{}   %Description of the instrument and its operation
%\Comm{}   %On which telescope(s) has instrument been commissioned/used
%\WV{}     %Total weight and value of equipment to be shipped
%\Wfocus{} %Weight at the focus (including ancillary equipment)
%\Interf{} %Compatibility of attachment interface with required focus
%\Focal{}  %Back focal distance value
%\Acqu{}   %Acquisition, focusing, and guiding procedure
%\Softw{}  %Compatibility with ESO software standards (data handling)
%\Suppl{}  %Estimate of services expected from ESO (in person days)

%%%%%%%%%%%%%%%%%%%%%%%%%%%%%%%%%%%%%%%%%%%%%%%%%%%%%%%%%%%%%%%%%%%%%%%%
%%%%% THE END %%%%%%%%%%%%%%%%%%%%%%%%%%%%%%%%%%%%%%%%%%%%%%%%%%%%%%%%%%
%%%%%%%%%%%%%%%%%%%%%%%%%%%%%%%%%%%%%%%%%%%%%%%%%%%%%%%%%%%%%%%%%%%%%%%%
\MakeProposal
\end{document}


